\documentclass{article}
\usepackage{fixltx2e}
\setlength{\parskip}{\baselineskip}
\title{The Knapsack Problem \\ \small \textit{A Survey of Solution Approaches}}
\date{\small \today}
\author{Sayak Biswas \\ \small UNIVERSITY \textit{of} \textbf{FLORIDA} \\ \small UFID: 54584911}
\begin{document}
\pagenumbering{gobble}
\maketitle
\begin{abstract}
This paper surveys existing literature for different approaches to solve the knapsack problem. The Knapsack Problem is a combinatorial optimization problem in which one has to maximize the profits gained by packing a set of objects in a knapsack without exceeding its capacity. The problem is \textbf{\textit{NP}}-complete, thus there is no known polynomial time algorithm for a large input.

Specifically, we take a look at the \textit{0-1 Knapsack Problem} and provide a qualitative comparison between the three well-known approaches towards solving the problem: brute-force, dynamic programming and branch \& bound algorithms.
\end{abstract}
\newpage

\section{Introduction}
The \textit{0-1 Knapsack Problem} is an optimization problem, which at a high level is to choose the most profitable subset from a collection of available items without overloading the knapsack. The problem is formally defined as follows:

Given a knapsack of maximum capacity \textit{C} and \textit{n} items each weighing \textit{w\textsubscript{i}} and with an associated profit of \textit{p\textsubscript{i}}, the \textit{Knapsack Problem} is to choose a subset of the items such that the below holds true:

maximize $\sum_{i=1}^{n}\textit{p\textsubscript{i}x\textsubscript{i}}$

on the condition $\sum_{i=1}^{n}\textit{w\textsubscript{i}x\textsubscript{i}} \leq \textit{C}, \textit{x\textsubscript{i}} \in \{0, 1\}, \textit{i} = 1,...,\textit{n}$

The two most common variations are the \textit{Unbounded} and \textit{Bounded} Knapsack Problems. 

\pagenumbering{arabic}
\end{document}